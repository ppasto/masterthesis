% LaTeX file for Chapter 03


\chapter{Results}

\section{Important questions when forecasting recruitment at the design-stage of a study}

By normal approximation to the Poisson distribution $C(t)\sim \textrm{N}(\mu=\lambda t, \sigma^2=(\lambda t)^2)$, we know that the probability of recruiting the desired $N$ participants is 0.5. Which means that the study has 50\% chance of obtaining the desired sample size in the suggested $T$ \citep{carter2004application}. We would also be assuming that the recruitment rate is constant over time.

In fact, we do not need normal approximation to see this. This can be shown with the Poisson distribution itself. We only need to specify the probability above $\lambda t$, for example, 0.5. For large $\lambda$, 50\% of the distribution will be below $\lambda t$. With $\lambda < 1$ this is no longer the case. 

This raises two questions which will be answered throughout this Master Thesis:
\begin{enumerate}
\item \textbf{Rate:} If $T$ is fixed, what does the expected rate $\lambda$ need to be to achieve a certain certainty of enrolling the total sample size $N$ within the time frame $T$?
\item \textbf{Time:} Given a certain rate $\lambda$, how long should the recruitment period $T$ be planned to give a confidence above 50\% of recruiting the total sample size $N$? In Machine Learning, this confidence is aimed at 80\%. In \cite{carter2004application}, at 90\%.
\end{enumerate}

\section{Pros and cons of Monte Carlo's simulations}

Carter suggests using Monte Carlo simulations, independent and identically distributed realizations of random variables. One clear advantage of MC simulations is their flexibility, as we can simulate any distribution we want. However, we must consider the following when we use MC simulations instead of the exact probability distribution:

\begin{itemize}
\item $M$, the number of simulations
\item Set a seed for computational reproducibility
\item Monte Carlo standard errors (MCse) of estimates based on MC simulations
\item Pseudo random numbers generated in R rely in the assumption that these pseudo random numbers are close to the true realizations of random variables \citep{held2014applied}
\end{itemize}
\section{Counts: Comparison exact vs Monte Carlo simulations}

Carter raises two important questions, enumerated in the previous section \citep{carter2004application, carter2005practical}. He suggests the use of Monte Carlo (MC) simulations for counts. Here we investigate the accuracy of his MC simulations by comparing them with exact distributions for accrual of counts introduced in Chapter 2.

As we saw in the previous chapter we can model the counts taking into account only the aleatory uncertainty using $C(t)\sim \textrm{Po}(\lambda t)$. Or, more realistically, take into consideration the fluctuation of $\Lambda$ over time using $C(t)\sim \textrm{Po}(\Lambda t)$ with $\Lambda\sim \textrm{G}(\alpha, \beta)$.

In Figures \ref{fig:3_1} and \ref{fig:3_2}, we can see how we need at least $M=10^4$ MC results of simulations to converge to the theoretical probability distribution discussed in Chapter 2.

% otherwise, show picture of code


% 
% lalalalal
% <<echo=TRUE, eval=TRUE>>=
% set.seed(2025)
% 
% M <- 10^5
% t <- seq(1, 550, 1)
% lambda <- 0.591
% 
% # Generate cumulative Poisson paths
% cval_cum_matrix <- matrix(NA, nrow = n, ncol = length(t))
% 
% for (i in 1:M) {
% 	cval <- rpois(length(t), lambda)
% 	cval_cum_matrix[i, ] <- cumsum(cval)
% }
% 
% # Extract final counts for histogram
% final_counts <- cval_cum_matrix[, length(t)]
% @
% 

\begin{figure}
\begin{knitrout}
\definecolor{shadecolor}{rgb}{0.969, 0.969, 0.969}\color{fgcolor}
\includegraphics[width=\maxwidth]{figure/ch03_figunnamed-chunk-1-1} 
\end{knitrout}
  \caption{Comparison of theoretical Probability Mass Function (PMF) of Poisson model for counts with $\lambda = 0.591$ for accrual at time $t=550$ and Monte Carlo (MC) simulations with $M=10^4$.}
  \label{fig:3_1}
\end{figure}



\begin{figure}
\begin{knitrout}
\definecolor{shadecolor}{rgb}{0.969, 0.969, 0.969}\color{fgcolor}
\includegraphics[width=\maxwidth]{figure/ch03_figunnamed-chunk-2-1} 
\end{knitrout}
\caption{Comparison of theoretical Probability Mass Function (PMF) of Poisson-Gamma model for counts with $\alpha = 32.4$ and $\beta = 54.8$ for accrual at time $t=550$, and Monte Carlo (MC) simulations with $M=10^4$.}
\label{fig:3_2}
\end{figure}

\begin{table}[h!]
\centering
\begin{tabular}{cccc}
 \textbf{Model} & \textbf{Estimated Probabilty} & \textbf{MCse} & \textbf{Exact Probability} \\
\hline
\hline
 $C(T)\sim\textrm{Po}(\lambda T)$ & $\textrm{P}(C(T)\geq 324) = 0.504$ & 0.71 & 0.508 \\
 $C(T)\sim\textrm{PoG}(T, \alpha, \beta)$ & $\textrm{P}(C(T)\geq 324) = 0.48$ & 0.693 & 0.501 
\end{tabular}
\caption{Probability estimates based on Monte Carlo simulations, their respective Monte Carlo standard errors (MCse) and exact probability computations for processes considered in modeling counts where the waiting time is fixed as $T = 548$,  $\lambda = 0.591$, $\alpha = 32.4$ and $\beta = 54.8$.}
\label{tab:mcsec}
\end{table}


\section{Time: Comparison exact vs Monte Carlo simulations}


As we saw in the previous chapter we can model the waiting time to accrue $c = 324$ objects when unit-time recruitment rate is $\lambda = 0.591$ taking into account only aleatory uncertainty with $T(c)\sim \textrm{Erlang}(c,\lambda)$. Or, taking into account both aleatory and epistemic uncertainty with $T(c)\sim\textrm{G}(c, \Lambda)$ where $\Lambda\sim \textrm{G}(\alpha,\beta)$.


In Figures \ref{fig:3_3} and \ref{fig:3_4}, we can see how we need at least $M=10^4$ MC results of simulations converge to the theoretical probability distribution discussed in Chapter 2.

% LALALA
% <<echo=TRUE, cache=TRUE, warning=FALSE>>=
% set.seed(2025)
% M <- 10^5
% Nneed <- 324
% lambda <- 0.591
% 
% timep <- rep(0, M)
% csump <- rep(0, M)
% 
% for(m in 1:M){
% 	while (csump[m] < Nneed) {
% 		csump[m] <- csump[m] + rpois(1, lambda)
% 		timep[m] <- timep[m] + 1
% 	}
% }
% @
% first check if it works!!
% add also the code chunk with negative binomial (extension of carter and bagiella)

\begin{figure}
\begin{knitrout}
\definecolor{shadecolor}{rgb}{0.969, 0.969, 0.969}\color{fgcolor}
\includegraphics[width=\maxwidth]{figure/ch03_figunnamed-chunk-3-1} 
\end{knitrout}
\caption{Comparison of theoretical density function of Erlang model for time with parameters $c = 324$ and $\lambda = 0.591$ and Monte Carlo (MC) simulations with $M=10^4$.}
\label{fig:3_3}
\end{figure}


\begin{figure}
\begin{knitrout}
\definecolor{shadecolor}{rgb}{0.969, 0.969, 0.969}\color{fgcolor}
\includegraphics[width=\maxwidth]{figure/ch03_figunnamed-chunk-4-1} 
\end{knitrout}
\caption{Comparison of theoretical Density function of Gamma-Gamma model for time with parameters $\alpha = 32.4$ and $\beta = 54.8$ and Monte Carlo (MC) simulations with $M=10^4$.}
\label{fig:3_4}
\end{figure}






\begin{table}[h!]
\centering
\begin{tabular}{cccc}
 \textbf{Model} & \textbf{Estimated Probabilty} & \textbf{MCse} & \textbf{Exact Probability} \\
\hline
\hline
 $T(N)\sim\textrm{G}(N, \lambda)$& $\textrm{P}(T(N)\geq 548) = 0.498$ & 0.706 & 0.496\\
$T(N)\sim\textrm{GG}(N, \alpha, \beta)$ & $\textrm{P}(T(N)\geq 548) = 0.52$ & 0.721 & 0.52
\end{tabular}
\caption{Probability estimates based on Monte Carlo simulations, their respective Monte Carlo standard errors (MCse) and exact probability computations for all processes considered in modeling time where the sample size is fixed to be $N = 324$,  $\lambda = 0.591$, $\alpha = 32.4$ and $\beta =54.8$.}
\label{tab:mcset}
\end{table}
% 
% <<>>=
% qgamma(p = 0.9, shape = 324, rate = 0.591)
% @

Small number of simulations (up to 1000) from \cite{carter2004application} gives him inaccurate estimate of days (580), when the exact using the Erlang distribution gives  588.

In Tables \ref{tab:mcsec} and \ref{tab:mcset} we illustrate how, indeed, by using the expected accrual without taking into consideration any uncertainty we have roughly a 50\% chance of accruing the desired sample size in the established time-frame \citep{carter2004application}.


